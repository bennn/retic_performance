Gradual typing promises to combine the benefits of dynamic and static typing
disciplines in a single language, giving developers freedom to use type information
only where it proves useful for catching bugs, documenting interfaces, or enabling optimizations.
Reticulated Python is one such \emph{gradually typed} language; it gives Python programmers
the ability to incrementally add sound, optional type annotations to critical parts of their programs.
Reticulated preserves type soundness by adding dynamic type tests to a program.
These tests can arbitrarily degrade the performance of programs, but their effect on
partially-typed variations of realistic programs is unknown.

This paper studies the overhead of gradual typing in Reticulated Python.
We examine $M$ programs taken from a variety of areas including Python libraries, implementations of common algorithms, and games.
For each program we systematically measure all partially-typed \emph{configurations} obtained by assigning full type signatures to a subset of functions in the program.
For a program with $n$  functions we obtain  $2^n$ configurations.
We additionally sample random configurations in which any subset of class fields, function parameters, or function return types may be typed.
Finally we attempt to explain which functions or types are responsible for the greatest overhead and generalize common patterns.






