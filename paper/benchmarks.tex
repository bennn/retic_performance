\section{Benchmarks}

\begin{figure}
  \begin{tabular}{$c@c@c@c@c}
    \rowstyle{\bfseries}
    Name & \# Mod & \# Func (\% typed) & \# Param \\ 
    Sample FSM & 5 & 999 (10\%)  & 999 \\
    List Interpreter & 34 & 999 & 999 \\
    Take5 & 3 & 999 & 999 \\
    Evolution & 77 & 999 & 999 \\
    Espionage & 2 & 999 & 999 \\
    Slow SHA & 4 & 999 & 999 \\
  \end{tabular}
  \caption{Benchmark Characteristics}
  \label{fig:benchmark}
\end{figure}

% remove BSL
% mention size of the reader, parser etc. - how much computation time does each part take to run
% for take5, talk about making the program loop to inc. time > 1 sec


 Figure~\ref{fig:benchmark} talks about the size of our benchmarks.

\bm{Sample FSM}
This program simulates the tension between individuals focused on their own interest, and the interest of the general population.
In this program, individuals are represented by finite state automaton. An automata describes how an individual acts. The transitions between
the different states of the automaton represent how the individual interacts with other individuals. This simulation assumes a fixed number of
actions. The program was fully annotated with types to obtain the various configurations.

\bm{Lisp Intepreter} 
This project implements an evaluator for a small subset of Racket, called Beginning Student Language (BSL).
The program receives a BSL program, evaluates it and prints the result. The program consists of a reader, a parser
and an interpreter, and overall, the program contains n functions which would produce $2^n$ configurations.
Since this number of configurations is too large to run, We chose to annotate  the reader  

\bm{Take5} 
This program is a simulation  of  a Take5 game. The game consists of a dealer
and a number of players. It  is played in rounds, each round consists of turns.
The game is played following a set of rules, and after each turn, the players'
scores are updated. At the end of the game the program outputs the final scores
This program was entirely typed. On average, a Take5 configuration runs in x
seconds. Since that time was too small to measure the differences in performance
across various configurations, The time was scaled by x by letting the program  run x consequtive  simulations
of the  game.


\bm{Evolution}
Simulation of an Evolution game. The program takes an argument n, representing the
number of players participating in the game. It then creates a dealer which runs a complete
simulation of the game. The program then outputs the players' scores.
Only three files from the program where typed. Those files were of the classes/types  for all
 card plays a player can make during a turn.


\bm{Espionage}
This program implements the minimum spanning tree algorithm using Kruskal's algorithm.
The program was entirely typed to produce the various configurations.

\bm{slowSHA}
Adapted from Vitousek et al.~\cite{vksb-dls-2014}. Originally from \url {http://github.com/sfstpala/SlowSHA}  Computes SHA-1  and SHA-512 for a sequence of
English words. 


